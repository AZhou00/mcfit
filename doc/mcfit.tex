\documentclass{article}
\usepackage{amsmath}
\usepackage{amssymb}
\usepackage[margin=2cm]{geometry}
\usepackage{hyperref}


\renewcommand{\d}{\mathrm{d}}
\newcommand{\xmin}{x_\textrm{min}}
\newcommand{\xmax}{x_\textrm{max}}
\newcommand{\ymin}{y_\textrm{min}}
\newcommand{\ymax}{y_\textrm{max}}
\newcommand{\Nhalf}{\lfloor N/2\rfloor} 
\newcommand{\Tophat}{\textrm{T}} 
\newcommand{\Gauss}{\textrm{G}} 


\begin{document}
\title{\texttt{mcfit}: Multiplicative Convolutional Fast Integral Transforms}
\author{Yin Li}
\date{Mar. 2017}
\maketitle


This package computes integral transforms with product kernels
using the FFTLog algorithm \cite{Talman78,Hamilton00}
\begin{equation}
    G(y) = \int_0^\infty F(x) K(xy) \,\frac{\d x}x
\end{equation}
or equivalently
\begin{equation}
    \label{IT}
    g(y) = \int_0^\infty f(x) (xy)^q K(xy) \,\frac{\d x}x
\end{equation}
where $K(xy)$ is the kernel function and $q$ is a power-law tilt parameter.
And with $f(x)\propto x^{-q}$, $g(y)\propto y^q$, one is free to choose $q$.
We can tilt $f(x)$ to balance $f$ at large and small $x$,
in order to improve accuracy and reduce ringing.

The idea is to take advantage of the convolution theorem in $\ln x$ and $\ln y$.
We approximate $f(x)$ with truncated Fourier series,
and use the exact Fourier transform of the possibly oscillatory kernel.
We can calculate the latter analytically via Mellin transform.
This algorithm performs the best if $f$ can be accurately approximated
by a sum of finite Fourier modes and logarithmically uniform discretization,
that is when $f(x)$ is smooth and spans a large range in $\ln x$.


\paragraph{Derivation} largely follows \cite{Hamilton00}.
In Eq.~\eqref{IT} take $f(x)$ on $[\xmin, \xmax]$, impose periodicity
$f(\xmax)=f(\xmin)$ in $\ln x$, and then discretize it by
$\ln x_n\equiv\ln\xmin+n\Delta$ and $\ln\xmax/\xmin=N\Delta$.
The logarithmic interval $\Delta\equiv|\Delta\ln x|\equiv|\Delta\ln y|$
uniformly discretizes $\ln x$ and later also $\ln y$,
i.e., $\ln y_n\equiv\ln\ymin+n\Delta$ and $\ln\ymax/\ymin=N\Delta$.

Define a symmetrized sum
\begin{equation}
    {\sum_n}' a_n \equiv \sum_{m=-\Nhalf}^{\Nhalf} w_n a_n
\end{equation}
with $w_n=1$ except that $w_{-N/2}=w_{N/2}=1/2$ if N is even.
Using this and the fact that $f_m$ and $f(x_n)$ are periodic over $m,n\in\mathbb{Z}_N$,
let's rewrite the usual discrete Fourier transform
\begin{align}
    f(x_n) &= \frac1N \sum_{m=0}^{N-1} f_m \exp\Bigl(\frac{i2\pi mn}N\Bigr)
    = \frac1N {\sum_m}' f_m \exp\Bigl(\frac{i2\pi mn}N\Bigr)
    = \frac1N {\sum_m}' f_m \exp\Bigl[\frac{i2\pi m\ln(x_n/\xmin)}{N\Delta}\Bigr]  \\
    \label{step1}
    f_m &= \sum_{n=0}^{N-1} f(x_n) \exp\Bigl(-\frac{i2\pi mn}N\Bigr)
    = {\sum_n}' f(x_n) \exp\Bigl(-\frac{i2\pi mn}N\Bigr)
\end{align}
For smooth $f(x)$, a good approximation is the truncated Fourier series
with $f_m$ as coefficients
\begin{equation}
    f(x) \approx \frac1N {\sum_m}' f_m \exp\Bigl[\frac{i2\pi m\ln(x/\xmin)}{N\Delta}\Bigr]
\end{equation}
always real thanks to the symmetrized sum.

To evaluate Eq.~\eqref{IT} we can transform each Fourier mode
\begin{align}
    &\int_0^\infty \exp\Bigl[\frac{i2\pi m\ln(x/\xmin)}{N\Delta}\Bigr]
        (xy)^q K(xy) \,\frac{\d x}x  \\
    = &\int_0^\infty \exp\biggl\{\frac{i2\pi m}{N\Delta}
        \Bigl[\ln(xy)-\ln\Bigl(\frac{y}{\ymin}\Bigr)
            +\ln\Bigl(\frac{\ymax}{\ymin}\Bigr)-\ln(\xmin\ymax)\Bigr]
    \biggr\} (xy)^q K(xy) \,\frac{\d x}x  \\
    = &\exp\biggl\{-\frac{i2\pi m}{N\Delta}
        \Bigl[\ln\Bigl(\frac{y}{\ymin}\Bigr)+\ln(\xmin\ymax)\Bigr]
    \biggr\} \int_0^\infty t^{q-1+i\tfrac{2\pi m}{N\Delta}} K(t) \,\d t
\end{align}
Finally we just need to calculate analytically
the Fourier transform of the kernel including the tilt factor,
as a Mellin transform
\begin{equation}
    \label{Mellin}
    U_K(z) \equiv \int_0^\infty t^{z-1} K(t) \,\d t
\end{equation}
which satisfies $U_K^*(z)=U_K(z^*)$, and derive
\begin{align}
    g(y) &\approx \frac1N {\sum_m}' g_m \exp\Bigl[-\frac{i2\pi m\ln(y/\ymin)}{N\Delta}\Bigr]  \\
    \label{step6}
    g(y_n) &\approx \frac1N {\sum_m}' g_m \exp\Bigl[-\frac{i2\pi m\ln(y_n/\ymin)}{N\Delta}\Bigr]
    = \frac1N \sum_{m=0}^{N-1} g_m \exp\Bigl(-\frac{i2\pi mn}N\Bigr)
\end{align}
where
\begin{align}
    \label{step5}
    g_m &= f_m u_m  \\
    \label{step4}
    u_m &= \exp\Bigl[-\frac{i2\pi m\ln(\xmin\ymax)}{N\Delta}\Bigr]
    U_K\Bigl(q+i\frac{2\pi m}{N\Delta}\Bigr) \qquad -\Nhalf\leq m\leq\Nhalf
\end{align}

We should make $u_m$ periodic over $m\in\mathbb{Z}_N$, however
$u_{-N/2}$ and $u_{N/2}$ are not necessarily equal for even $N$.
In practice we can impose
\begin{equation}
    \label{step3nolowring}
    g_{\pm N/2} \to \Re(g_{N/2}) \qquad N\in2\mathbb{Z}
\end{equation}
which is automatically satisfied if
\begin{equation}
    \label{step3lowring}
    \ln(\xmin\ymax) = \frac\Delta\pi \mathrm{Arg} U_K\Bigl(q+i\frac\pi\Delta\Bigr)
    + \textrm{integer} \cdot \Delta
\end{equation}
where the argument takes its principal value in $(-\pi,\pi]$
and we set the arbitrary integer to 0.
Otherwise we can just set $\xmin\ymax=1$ for convenience.


\paragraph{Algorithm}
Follow the steps in equations \eqref{step1}, \eqref{Mellin},
\eqref{step3lowring} (or \eqref{step3nolowring}), \eqref{step4}, \eqref{step5},
and \eqref{step6} for fast evaluation of Eq.~\eqref{IT}.
$u_m$ from Eq.~\eqref{step4} should be saved for multiple transformations.


\paragraph{Common Integral Transforms:}


\subparagraph{Hankel transform pair}
\begin{equation}
    G(y) = \int_0^\infty F(x) J_\nu(xy) \,x\d x
\end{equation}
\begin{align}
    f(x) &\to x^{2-q}F(x)  \\
    g(y) &\to y^q G(y)  \\
    K(t) &\to J_\nu(t)  \\
    U_K(z) &= 2^{z-1} \frac{\Gamma\bigl(\frac{\nu+z}2\bigr)}{\Gamma\bigl(\frac{2+\nu-z}2\bigr)}
\end{align}
The pair is symmetric when $q=1$.


\subparagraph{Spherical Bessel transform pair}
\begin{equation}
    G(y) = \sqrt\frac2\pi \int_0^\infty F(x) j_\nu(xy) \,x^2\d x
\end{equation}
\begin{align}
    f(x) &\to x^{3-q} F(x)  \\
    g(y) &\to y^q G(y)  \\
    K(t) &\to \sqrt\frac2\pi j_\nu(t) = \frac{J_{\nu+\frac12}(t)}{\sqrt t}  \\
    U_K(z) &= 2^{z-\frac32} \frac{\Gamma\bigl(\frac{\nu+z}2\bigr)}{\Gamma\bigl(\frac{3+\nu-z}2\bigr)}
\end{align}
The pair is symmetric when $q=\frac32$.


\subparagraph{Fourier sine transform pair}
\begin{equation}
    G(y) = \sqrt\frac2\pi \int_0^\infty F(x) \sin(xy) \,\d x
\end{equation}
\begin{align}
    f(x) &\to x^{1-q} F(x)  \\
    g(y) &\to y^q G(y)  \\
    K(t) &\to \sqrt\frac2\pi \sin(t) = \sqrt t J_\frac12(t)  \\
    U_K(z) &= \sqrt\frac2\pi \Gamma(z) \sin\bigl(\frac{\pi z}2\bigr)
    = 2^{z-\frac12} \frac{\Gamma\bigl(\frac{1+z}2\bigr)}{\Gamma\bigl(\frac{2-z}2\bigr)}
\end{align}
The pair is symmetric when $q=\frac12$.


\subparagraph{Fourier cosine transform pair}
\begin{equation}
    G(y) = \sqrt\frac2\pi \int_0^\infty F(x) \cos(xy) \,\d x
\end{equation}
\begin{align}
    f(x) &\to x^{1-q} F(x)  \\
    g(y) &\to y^q G(y)  \\
    K(t) &\to \sqrt\frac2\pi \cos(t) = \sqrt t J_{-\frac12}(t)  \\
    U_K(z) &= \sqrt\frac2\pi \Gamma(z) \cos\bigl(\frac{\pi z}2\bigr)
    = 2^{z-\frac12} \frac{\Gamma\bigl(\frac{z}2\bigr)}{\Gamma\bigl(\frac{1-z}2\bigr)}
\end{align}
The pair is symmetric when $q=\frac12$.


\paragraph{Applications}


\subparagraph{Top-hat smoothing of a radial function}
in $\mathbb{R}^d$
\begin{equation}
    F_W(R) = \int_0^\infty \frac{k^dF(k)}{2^{d-1}\pi^\frac{d}2\Gamma\bigl(\frac{d}2\bigr)}
                    W_\Tophat(kR) \,\frac{\d k}k
\end{equation}
where
\begin{align}
    W_\Tophat(r) &= \begin{cases}
        1 & r\leq R \\
        0 & r > R
    \end{cases} \\
    W_\Tophat(kR) &= \frac{2^\frac{d}2\Gamma\bigl(\frac{d}2+1\bigr)J_\frac{d}2(kR)}{(kR)^{d/2}}
\end{align}
\begin{align}
    x &\to k  \\
    y &\to R  \\
    f(x) &\to \frac{k^{d-q}F(k)}{2^{d-1}\pi^\frac{d}2\Gamma\bigl(\frac{d}2\bigr)}  \\
    g(y) &\to R^q F_W(R)  \\
    K(t) &\to W_\Tophat(t)  \\
    U_K(z) &= 2^{z-1} \frac{\Gamma\bigl(\frac{2+d}2\bigr)\Gamma\bigl(\frac{z}2\bigr)}
                        {\Gamma\bigl(\frac{2+d-z}2\bigr)}
\end{align}


\subparagraph{Gaussian smoothing of a radial function}
in $\mathbb{R}^d$
\begin{equation}
    F_W(R) = \int_0^\infty \frac{k^dF(k)}{2^{d-1}\pi^\frac{d}2\Gamma\bigl(\frac{d}2\bigr)}
                    W_\Gauss(kR) \,\frac{\d k}k
\end{equation}
where
\begin{align}
    W_\Gauss(r) &= \frac1{(2\pi R^2)^\frac{d}2} \exp\Bigl(-\frac{r^2}{2R^2}\Bigr)  \\
    W_\Gauss(kR) &= \exp\Bigl(-\frac{k^2R^2}2\Bigr)
\end{align}
\begin{align}
    x &\to k  \\
    y &\to R  \\
    f(x) &\to \frac{k^{d-q}F(k)}{2^{d-1}\pi^\frac{d}2\Gamma\bigl(\frac{d}2\bigr)}  \\
    g(y) &\to R^q F_W(R)  \\
    K(t) &\to W_\Gauss(t)  \\
    U_K(z) &= 2^{\frac{z}2-1} \Gamma\bigl(\tfrac{z}2\bigr)
\end{align}


\paragraph{Cosmology applications}


\subparagraph{Power spectrum to correlation function}
\begin{equation}
    \xi_l(r) = i^l \int_0^\infty \frac{k^3P_l(k)}{2\pi^2} j_l(kr) \,\frac{\d k}k
\end{equation}
\begin{align}
    x &\to k  \\
    y &\to r  \\
    f(x) &\to \frac{i^l k^{3-q} P_l(k)}{(2\pi)^\frac32}  \\
    g(y) &\to r^q \xi_l(r)
\end{align}
with the rest the same as the spherical Bessel transform.


\subparagraph{Correlation function to power spectrum}
also radial profile to its Fourier transform
\begin{equation}
    P_l(k) = (-i)^l \int_0^\infty 4\pi r^3\xi_l(r) j_l(kr) \,\frac{\d r}r
\end{equation}
\begin{align}
    x &\to r  \\
    y &\to k  \\
    f(x) &\to r^{3-q} \xi_l(r)  \\
    g(y) &\to \frac{i^l k^q P_l(k)}{(2\pi)^\frac32}
\end{align}
with the rest the same as the spherical Bessel transform.
The 2-pt function conversion pair is symmetric when $q=\frac32$.


\subparagraph{Variance in a Top-hat window}
in $\mathbb{R}^3$
\begin{equation}
    \sigma_\Tophat^2(R) = \int_0^\infty \frac{k^3P(k)}{2\pi^2} W_\Tophat^2(kR) \,\frac{\d k}k
\end{equation}
where
\begin{equation}
    W_\Tophat(t) = \frac3{t^3}(\sin t - t\cos t) = \frac{3j_1(t)}t
\end{equation}
\begin{align}
    x &\to k  \\
    y &\to R  \\
    f(x) &\to \frac{k^{3-q}P(k)}{2\pi^2}  \\
    g(y) &\to R^q \sigma_\Tophat^2(R)  \\
    K(t) &\to W_\Tophat^2(t)  \\
    U_K(z) &= \frac{9\sqrt{\pi}(z-2)}{4(z-6)}
        \frac{\Gamma\bigl(\frac{z-4}2\bigr)}{\Gamma\bigl(\frac{5-z}2\bigr)}
\end{align}


\subparagraph{Variance in a Gaussian window}
in $\mathbb{R}^3$
\begin{equation}
    \sigma_\Gauss^2(R) = \int_0^\infty \frac{k^3P(k)}{2\pi^2} W_\Gauss^2(kR) \,\frac{\d k}k
\end{equation}
where
\begin{equation}
    W_\Gauss(t) = \exp\Bigl(-\frac{t^2}2\Bigr)
\end{equation}
\begin{align}
    x &\to k  \\
    y &\to R  \\
    f(x) &\to \frac{k^{3-q}P(k)}{2\pi^2}  \\
    g(y) &\to R^q \sigma_\Gauss^2(R)  \\
    K(t) &\to W_\Gauss^2(t)  \\
    U_K(z) &= \frac12 \Gamma\bigl(\frac{z}2\bigr)
\end{align}


\subparagraph{Excursion set}
in $\mathbb{R}^3$
\begin{equation}
    \delta^W(R) = \int_0^\infty \frac{k^3\delta(k)}{2\pi^2} W_\Tophat(kR) \,\frac{\d k}k
\end{equation}
same as top-hat smoothing of a radial function given previously.
\begin{align}
    x &\to k  \\
    y &\to R  \\
    f(x) &\to \frac{k^{3-q}\delta(k)}{2\pi^2}  \\
    g(y) &\to R^q \delta^W(R)  \\
    K(t) &\to W_\Tophat(t)  \\
    U_K(z) &= 3\sqrt\pi 2^{z-3}
            \frac{\Gamma\bigl(\frac{z}2\bigr)}{\Gamma\bigl(\frac{5-z}2\bigr)}
\end{align}
Note that $\delta(k)$ is random and not smooth.


\paragraph{Test}
We test our implementation on Hankel transform in the following analytic case
%\begin{equation}
%    \int_0^\infty x^p e^{-x} J_\mu(xy) \,y\d x
%    = \frac{y^{1+\mu}}{2^\mu} \frac{\Gamma(1+\mu+p)}{\Gamma(1+\mu)}
%    \,_2F_1\bigl(\frac{1+\mu+p}2, \frac{2+\mu+p}2, 1+\mu; -y^2\bigr)
%\end{equation}
\begin{align}
    e^{-y} &= \int_0^\infty (1+x^2)^\frac32 J_0(xy) \,x\d x  \\
    (1+y^2)^\frac32 &= \int_0^\infty e^{-y} J_0(xy) \,x\d x
\end{align}


\paragraph{Parameters}


\subparagraph{Length: $N$}
Odd $N$ has worse ringing problem, similar to the case without low-ringing condition.


\subparagraph{Tilt: $q$}
We can reduce ringing by balancing the large and small scale amplitudes with the tilt.
The gamma function has simple poles at the non-positive integers,
which can be avoided by picking $q$.
However, one should not use a tilt too large, as it tends to shift the focus away
and thus reduces the accuracy on the interested scales.


\subparagraph{Padding:}
$f(x)$ is padded either with power-law extrapolations from the end segments
beyond the tabulated region, or just with zeros.


\paragraph{Gamma function}
The reflection and duplication formulae are useful for playing with the gamma functions
\begin{align}
    \frac\pi{\sin\pi z} &= \Gamma(1-z)\Gamma(z) = (-1)^n\Gamma(1+n-z)\Gamma(z-n)  \\
    \Gamma(z)\Gamma(z+\tfrac12) &= 2^{1-2z}\sqrt\pi\,\Gamma(2z)
\end{align}


\bibliographystyle{alpha}
\bibliography{mcfit.bib}
\end{document}
